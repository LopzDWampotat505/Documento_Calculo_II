%%%%%%%%%%%%%%%%%%%%%%%%%%%%%%%%%%%%%%%%%%%%%%%%%%%%%%%%%%%%%%%%%%%%%%%%%
%%%% Plantilla realizada por César Martínez cesar.martinez@udlap.mx %%%%%
%%%%%%%%%%%%%%%%%%%%%%%%%%%%VERSION 1.0%%%%%%%%%%%%%%%%%%%%%%%%%%%%%%%%%%
%%%%%%%%%%%%%%%%%%%%%%%%%%%%%%%%%%%%%%%%%%%%%%%%%%%%%%%%%%%%%%%%%%%%%%%%%
%%%%%%%%%%%%%%%%%%%%%%%%%%%%%%%%%%%%%%%%%%%%%%%%%%%%%%%%%%%%%%%%%%%%%%%%%


\documentclass[12pt]{article}  %tipo de documento y tamaño de letra normal
%%%%%%%%%%%%%%%%%%%%%%%%%%%%%%%%%%%%%%%%%%%%%%%%%%%%%%%%%%%%%%%%%%%%%
%%%%%%%%%%%%%%%%%%%%%%%%%%%%%%%%%%%%%%%%%%%%%%%%%%%%%%%%%%%%%%%%%%%%%
%%%%%%%%%%%%%%%%%%%%%%%%%%%%%%%%%%%%%%%%%%%%%%%%%%%%%%%%%%%%%%%%%%%%%
%%%%%%%% Paquetes basicos, pueden encontrar información especifica de cada uno de ellos en  (https://www.ctan.org/) %%%%%%%%%%%%%%%%%%%%%%% %%%%%%%%%%%%%%%%%%%%%%%%%%%%%%%%%%%%%%%%%%%%%%%%%%%%%%%%%%%%%%%%%%%%%
%%%%%%%%%%%%%%%%%%%%%%%%%%%%%%%%%%%%%%%%%%%%%%%%%%%%%%%%%%%%%%%%%%%%%
%%%%%%%%%%%%%%%%%%%%%%%%%%%%%%%%%%%%%%%%%%%%%%%%%%%%%%%%%%%%%%%%%%%%%
%\usepackage[spanish]{babel} %Indica que escribiermos en español
\usepackage[english]{babel} %Indica que escribiermos en inglés
%Comentar la línea del idioma que NO usarán en su reporte
\usepackage[utf8]{inputenc} %Indica qué codificación se está usando ISO-8859-1(latin1)  o utf8  
\usepackage{amsmath} % Comandos extras para matemáticas (cajas para ecuaciones,etc)
\usepackage{amssymb} % Simbolos matematicos (por lo tanto)
\usepackage{graphicx} % Incluir imágenes en LaTeX
\usepackage{color} % Para colorear texto
\usepackage{subfigure} % subfiguras
\usepackage{enumerate} % enumerar
\usepackage{commath} % funcionalidades extras para diferenciales, integrales,etc (\od, \dif, etc)
\usepackage{cancel} % para cancelar expresiones (\cancelto{0}{x})
\usepackage{float} %Podemos usar el especificador [H] en las figuras para que se queden donde queramos
\usepackage{appendix} %Para crear apendices
\usepackage{xcolor} %Definir colores personalizados
%%%%%%%%%%%%%%%%%%%%%%%%%%%%%%%%%%%%%%%%%%%%%%%%%%%%%%%%%%%%%%%%%%%%%
%%%%%%%%% PAQUETES CON OPCIONES ESPECIFICAS PRECARGADAS %%%%%%%%%%%%%
%%%%%%%%%%%%%%%%%%%%%%%%%%%%%%%%%%%%%%%%%%%%%%%%%%%%%%%%%%%%%%%%%%%%%
%%%%%%%%%%%%% Permitir agregar código, colocarlo en un rectángulo y  numerarlo %%%%%%%%%%%%%%%%%%%%%%%%%%%%%%%%%%%%%%%%%%%%%%%%%%%%%%%%%%%
%%%%%%%%%%%%%%%%%%%%%%%%%%%%%%%%%%%%%%%%%%%%%%%%%%%%%%%%%%%%%%%%%%%%%
%%%%%%%%%%%%%%%%%%%%%%%%%%%%%%%%%%%%%%%%%%%%%%%%%%%%%%%%%%%%%%%%%%%%%%%%%%%%%%%%%%%%%%%%%%%%%%%%%%%%%%%%%%%%%%%%%%%%%%%%%%%%%%%%%%%%%%%%%%
\usepackage{listings} %Sirve para pegar codigo fuente de programas
\usepackage{caption} %Agregar titulos a los codigos
\DeclareCaptionFont{white}{\color{white}}
\DeclareCaptionFormat{listing}{%
  \parbox{\textwidth}{\colorbox{gray}{\parbox{\textwidth}{#1#2#3}}\vskip-4pt}}
\captionsetup[lstlisting]{format=listing,labelfont=white,textfont=white}
\lstset{frame=lrb,xleftmargin=\fboxsep,xrightmargin=-\fboxsep}
\renewcommand{\lstlistingname}{Código}
%%%%%%%%%%%%%%%%%%%%%%%%%%%%%%%%%%%%%%%%%%%%%%%%%%%%%%%%%%%%%%%%%%%%%
%%% Definir márgenes del documento%%%%%%%%%%%%%%%%%%%%%%%%%%%%%%%%%%%
%%%%%%%%%%%%%%%%%%%%%%%%%%%%%%%%%%%%%%%%%%%%%%%%%%%%%%%%%%%%%%%%%%%%%
 \usepackage{anysize} % Para personalizar el ancho de  los márgenes
\marginsize{2cm}{2cm}{2cm}{2cm} % Izquierda, derecha, arriba, abajo
%%%%%%%%%%%%%%%%%%%%%%%%%%%%%%%%%%%%%%%%%%%%%%%%%%%%%%%%%%%%%%%%%%%%
%%% Hipervinculos activos y a color %%%%%%%%%%%%%%%%%%%%%%%%%%%%%%%%
%%%%%%%%%%%%%%%%%%%%%%%%%%%%%%%%%%%%%%%%%%%%%%%%%%%%%%%%%%%%%%%%%%%%%
\usepackage[colorlinks=true,plainpages=true,citecolor=blue,linkcolor=black]{hyperref}
\usepackage{hyperref} 
%%%%%%%%%%%%%%%%%%%%%%%%%%%%%%%%%%%%%%%%%%%%%%%%%%%%%%%%%%%%%%%%%%%%%
%%%%%% Encabezado y pie de pagina %%%%%%%%%%%%%%%%%%%%%%%%%%%%%%%%%%%
%%%%%%%%%%%%%%%%%%%%%%%%%%%%%%%%%%%%%%%%%%%%%%%%%%%%%%%%%%%%%%%%%%%%%
\usepackage{fancyhdr} 
\pagestyle{fancy}
\fancyhf{}
\fancyhead[L]{\footnotesize UDLAP} %encabezado izquierda
\fancyhead[R]{\footnotesize CEM}   % encabezado derecha
\fancyfoot[R]{\footnotesize \curso}  % Pie derecha
\fancyfoot[C]{\thepage}  % centro
\fancyfoot[L]{}  %izquierda
\renewcommand{\footrulewidth}{0.4pt}
%%%%%%%%%%%%%%%%%%%%%%%%%%%%%%%%%%%%%%%%%%%%%%%%%%%%%%%%%%%%%%%%%%%%%
%%%% Carpeta donde se deben colocar las imagenes %%%%%%%%%%%%%%%%%%%%
\graphicspath{{Imagenes/}} %Colocar aqui todas las imagenes del documento pueden estar en formato png, eps o jpg, se recomienda eps para mayor calidad.
%%%%%%%%%%%%%%%%%%%%%%%%%%%%%%%%%%%%%%%%%%%%%%%%%%%%%%%%%%%%%%%%%%%%%
%%%%%%%%%%%%%%%%%%%%%%%%%%%%%%%%%%%%%%%%%%%%%%%%%%%%%%%%%%%%%%%%%%%%%
%%%%%%%% Termina carga de paquetes %%%%%%%%%%%%%%%%%%%%%%%%%%%%%%%%%%% 
%%%%%%%%%%%%%%%%%%%%%%%%%%%%%%%%%%%%%%%%%%%%%%%%%%%%%%%%%%%%%%%%%%%%%
%%%%%%%%%%%%%%%%%%%%%%%%%%%%%%%%%%%%%%%%%%%%%%%%%%%%%%%%%%%%%%%%%%%%%
%%%%%%%%%%%%%%%%%%%%%%%%%%%%%%%%%%%%%%%%%%%%%%%%%%%%%%%%%%%%%%%%%%%%%
%%%%%% Modificar campos que aparecerán en portada %%%%%%%%%%%%%%%%%%%
%%%%%%%%%%%%%%%%%%%%%%%%%%%%%%%%%%%%%%%%%%%%%%%%%%%%%%%%%%%%%%%%%%%%%
%%%%%%%%%%%%%%%%%%%%%%%%%%%%%%%%%%%%%%%%%%%%%%%%%%%%%%%%%%%%%%%%%%%%%
%%%%%%%%%%%%%%%%%%%%%%%%%%%%%%%%%%%%%%%%%%%%%%%%%%%%%%%%%%%%%%%%%%%%%
\def\titulo{Reporte: Explorando la Optimización}% Título
\def\materia{Cálculo II MAT2012-13} %Clave nombre de la materia y sección
\def\prof{Anabel Hernandez Ramirez}
\def\curso{Cálculo II} %Nombre de la materia para footnote
\def\fecha{21 de Noviembre de 2025} % Fecha
\def\equipo {Malcolm in The Middle}%Verificar en blackboard el número asignado
\def\ida{183339} %Estudiante A
\def\esta{Juan Pablo Lopez Moreno}
\def\lica{LMT}

\def\idb{185406} %Estudiante B
\def\estb{Amy Marianee Ramírez Sánchez}
\def\licb{LMT}

\def\idc{183887} %Estudiante C
\def\estc{Malcolm Arriaga Galván}
\def\licc{LBM}

\def\idd{182733} %Estudiante D
\def\estd{Fátima Lucía Carmona Lagarda}
\def\licd{LQI}

\def\ide{186295} %Estudiante E
\def\este{Mayela Martin Meneses}
\def\lice{LQI}

\def\idf{182884} %Estudiante F
\def\estf{Yahel Reyes Noguez}
\def\licf{LBM}
\begin{document} %Inicia el documento
%%%%%%%%%%%%%%%%%%%%%%%%%%%%%%%%%%%%%%%%%%%%%%%%%%%%%%%%%%%%%%%%%%%%%
%%%%%%%%%%%%%%%%%%%%%%%%%%%%%%%%%%%%%%%%%%%%%%%%%%%%%%%%%%%%%%%%%%%%%
%%%%%%%%%%%%%%%%%%%%%%%%%%%%%%%%%%%%%%%%%%%%%%%%%%%%%%%%%%%%%%%%%%%%%
%%%%%%%%%%%%%%%%%%%%%%%%%%%%%%%%%% PORTADA %%%%%%%%%%%%%%%%%%%%%%%%%%
%%%%%%%%%%%%%%%%%%%%%%%%%%%%%%%%%%%%%%%%%%%%%%%%%%%%%%%%%%%%%%%%%%%%%
\begin{center} 
\newcommand{\HRule}{\rule{\linewidth}{0.5mm}} 
\thispagestyle{empty} 
\vspace*{-1.5cm} 
\textsc{\huge Universidad de las Américas Puebla}\\[1.5cm] 
\textsc{\LARGE Escuela de ciencias}\\[1.5cm] 
\textsc{\LARGE Departamento de Actuaría, Física y Matemáticas}\\[1.5cm]  
\includegraphics[width=150mm]{UDLAP}\vspace*{1cm}  \HRule \\[0.4cm] 
{ \huge \bfseries \titulo}\\[0.4cm] 
\HRule \\[1cm] 
{ \Large \bfseries \materia}\\ [0.25cm]
{ \Large \prof}\\[1cm]
{ \Large \bfseries Equipo: \equipo}\\[1cm] 
\begin{flushleft} \Large 
\ida \hspace{0.5cm}\esta \hspace{0.5cm} \lica\\
\idb \hspace{0.5cm}\estb \hspace{0.5cm} \licb\\
\idc \hspace{0.5cm}\estc \hspace{0.5cm} \licc\\
\idd \hspace{0.5cm}\estd \hspace{0.5cm} \licd\\
\ide \hspace{0.5cm}\este \hspace{0.5cm} \lice\\
\idf \hspace{0.5cm}\estf \hspace{0.5cm} \licf\\
%Copiar y pegar más líneas si su equipo tiene más de 2 integrantes, eliminar si la entrega es individual
\end{flushleft} 
\vfill 
\begin{center} 
{\Large \fecha, San Andrés Cholula, Puebla} 
\end{center} 
\end{center} \newpage 
%%%%%%%%%%%%%%%%%%%% TERMINA PORTADA %%%%%%%%%%%%%%%%%%%%%%%%%%%%%%%%
%%%%%%%%%%%%%%%%%%%%%%%%%%%%%%%%%%%%%%%%%%%%%%%%%%%%%%%%%%%%%%%%%%%%%
\setcounter{page}{1} %Para comenzar a numerar las páginas desde este punto

\section{Planteamiento del Problema}

La optimización con restricciones en funciones de dos variables es una técnica clave del análisis matemático, mediante la cual se busca optimizar (maximizar o minimizar) una función objetivo respetando ciertas condiciones que limitan las variables, como por ejemplo un volumen fijo o un presupuesto limitado \cite{IBM}.
Es relevante mencionar que este enfoque se desarrolla ampliamente en el libro: \textit{Cálculo de varias variables trascendentes tempranas}\cite{Stewart} y se aborda de forma más detallada en el libro: \textit{Engineering Design Optimization} \cite{Martins}. En particular, el capítulo \textit{Constrained Gradient-Based Optimization} explora a fondo los fundamentos teóricos necesarios para comprender este tipo de problemas.

\subsection*{Problema Específico}
Se desea diseñar un cilindro abierto (sin tapa) con volumen fijo de $V=500\text{ cm}^{3}$ de manera que se minimice el área total del material utilizado.

\begin{itemize}
 \item \textbf{Función Objetivo (Área a Minimizar):} $A(r,h)=\pi r^{2}+2\pi rh$.
 \item \textbf{Restricción (Volumen Fijo):} $V(r,h)=\pi r^{2}h = 500$.
\end{itemize}

\section{Método de Solución}

Para resolver el problema de optimización, se siguió el siguiente procedimiento:

\begin{enumerate}
 \item \textbf{Planteamiento de la Restricción:} Se despejó la altura $h$ en función del radio $r$ a partir de la restricción del volumen fijo $V=500\text{ cm}^3$.
 \item \textbf{Formulación del Área Total $A(r)$:} Se sustituyó la expresión de $h(r)$ en la función de área $A(r,h)$ para obtener una función de una sola variable, $A(r)$.
 \item \textbf{Búsqueda de Puntos Críticos:} Se calculó la primera derivada $A^{\prime}(r)$, se igualó a cero para encontrar el punto crítico, y se despejó el radio óptimo $r$.
 \item \textbf{Comprobación de Cálculos:} Se verificaron los datos obtenidos mediante \textit{Wolfram Alpha}.
 \item \textbf{Cálculo de la Altura Óptima:} El valor de $r$ hallado se sustituyó en la expresión de $h(r)$ para obtener la altura óptima $h$.
 \item \textbf{Comprobación (Clasificación):} Se utilizó la segunda derivada $A^{\prime\prime}(r)$ para verificar que el punto crítico corresponde a un mínimo.
 \item \textbf{Evaluación del Área Mínima:} Se sustituyó el radio óptimo en $A(r)$ para obtener el área mínima.
\end{enumerate}

\section{Cálculos Completos}

\subsection{Expresión del Área en una Variable}
\textbf{Restricción (Volumen):} $V=\pi r^{2}h=500$.
\begin{equation}
 h = \frac{500}{\pi r^{2}} \label{eq:h}
\end{equation}
\textbf{Función de Área:} $A(r,h)=2\pi rh+\pi r^{2}$.
Sustituyendo $h$ (Ec. \ref{eq:h}) en $A(r,h)$:
\begin{equation}
 A(r) = 2\pi r \left( \frac{500}{\pi r^{2}} \right) + \pi r^{2} = \frac{1000}{r} + \pi r^{2}
\end{equation}

\subsection{Puntos Críticos}
Calculando la primera derivada $A^{\prime}(r)$:
\begin{equation}
 A^{\prime}(r) = -\frac{1000}{r^{2}} + 2\pi r
\end{equation}
Igualando a cero para encontrar el punto crítico $A^{\prime}(r)=0$:
\begin{align*}
 -\frac{1000}{r^{2}} + 2\pi r &= 0 \\
 2\pi r &= \frac{1000}{r^{2}} \\
 2\pi r^{3} &= 1000 \\
 r^{3} &= \frac{1000}{2\pi}
\end{align*}
El radio óptimo es:
\begin{equation}
 r = \left(\frac{1000}{2\pi}\right)^{1/3} \approx 5.419\text{ cm}
\end{equation}

\subsection{Altura Óptima}
Sustituyendo $r$ en la ecuación de $h$ (Ec. \ref{eq:h}):
\begin{equation}
 h = \frac{500}{\pi (5.419)^{2}} \approx 5.419\text{ cm}
\end{equation}
\textbf{Resultado:} Se confirma que $r=h \approx 5.419\text{ cm}$.

\section{Clasificación de Puntos Críticos}

Para clasificar el punto crítico, se utiliza el criterio de la segunda derivada.

\subsection{Segunda Derivada}
Calculando la segunda derivada $A^{\prime\prime}(r)$:
\begin{equation}
 A^{\prime\prime}(r) = \frac{2000}{r^{3}} + 2\pi
\end{equation}

\subsection{Evaluación en el Punto Crítico}
Evaluando en el punto crítico (usando la aproximación $r \approx 5\text{ cm}$ para simplificar el cálculo en el reporte):
\begin{align*}
 A^{\prime\prime}(5) &= \frac{2000}{5^{3}} + 2\pi \\
 A^{\prime\prime}(5) &= \frac{2000}{125} + 2\pi \\
 A^{\prime\prime}(5) &= 16 + 2\pi
\end{align*}

Dado que $A^{\prime\prime}(5) > 0$, se concluye que el punto crítico corresponde a un \textbf{mínimo}. Esto significa que las dimensiones $r=h=5.419\text{ cm}$ minimizan el área total del material.

\section{Gráficas Generadas por el Equipo}
En primer lugar, para visualizar las soluciones del problema de optimización, se desarrolló una gráfica en Wolfram Alpha que representa la obtención de los puntos críticos de la función.
\begin{figure}[H]
 \centering
 \includegraphics[width=0.7\textwidth]{Wolfram Alpha.jpg}
 \caption{Gráfica generada en Wolfram Alpha para encontrar los puntos críticos de la función $A(r)$.}
\end{figure}
La gráfica muestra el modelo geométrico del cilindro sin tapa que se utiliza en el problema de optimización. En este modelo, el volumen del cilindro se mantiene fijo en $500\text{ cm}^3$, y el objetivo es encontrar las dimensiones (radio $r$ y altura $h$) que minimicen el área total del material necesario para construirlo.


\begin{figure}[H]
 \centering
 \includegraphics[width=0.3\textwidth]{Cilindro.png} 
 \caption{Representación Geométrica del Cilindro Óptimo.}
\end{figure}

La visualización ayuda a entender por qué la solución matemática exige que el radio y la altura deben ser iguales ($r = h$) para reducir al mínimo el material utilizado.
Asimismo, para apoyar en la visualización y demsotracion, se puede desarrollar un modelo en 3D, el cuál, mediante impresion 3D, se peude transmitir a un modelo exacto para mediciones físicas, demostrando el volumen adquirido.
\begin{figure}[H]
 \centering
 \includegraphics[width=0.5\textwidth]{Cilindro en CAD.png} 
 \caption{Modelo 3D del Cilindro Óptimo, diseñado mediante CAD (específicamente SolidWorks).}
\end{figure}

\section{Conclusiones}

Los resultados que obtuvimos muestran que el \textbf{área mínima del cilindro} ocurre cuando el radio y la altura son iguales, siendo estos valores aproximados $r=h \approx 5.41926\text{ cm}$. En estas dimensiones, el área total del material alcanza aproximadamente $A \approx 276.791\text{ cm}^{2}$. Esta solución evidencia cómo la \textbf{geometría óptima no surge por intuición}, sino del \textbf{análisis matemático riguroso} basado en derivadas y restricciones.

Algunos ejemplos en la vida real en los que se utiliza la \textbf{optimización con restricciones} podrían ser en los campos de \textbf{diseño industrial, la ingeniería y la manufactura}, donde reducir el área de material implica disminuir costos, peso y desperdicio. También se podría emplear para \textbf{optimizar envases, recipientes, estructuras, piezas mecánicas o cualquier objeto} cuya eficiencia dependa de su forma.
\begin{itemize}
 \item \textbf{Resultados:} El área mínima del cilindro ocurre cuando el radio y la altura son iguales, con valores aproximados $r=h \approx 5.41926\text{ cm}$.
 \item \textbf{Área Mínima:} En estas dimensiones, el área total del material alcanza aproximadamente $A \approx 276.791\text{ cm}^{2}$.
 \item \textbf{Aplicación Práctica:} Esta metodología se utiliza en campos como el diseño industrial, la ingeniería y la manufactura, donde reducir el área de material implica disminuir costos, peso y desperdicio, siendo esencial para optimizar envases y recipientes.
\end{itemize}

\section{Bibliografía}
\begin{thebibliography}{99}
 \bibitem{IBM} IBM. (s.f.). \textit{¿Qué es el modelado de optimización?} \url{https://www.ibm.com/mx-es/think/topics/optimization-model}.
 \bibitem{Martins} Martins, J. R. R. A., \& Ning, A. (2021). \textit{Engineering Design Optimization}. Cambridge University Press. \url{https://flowlab.groups.et.byu.net/mdobook.pdf}.
 \bibitem{LinaM3} La Prof Lina M3. (2019, 8 mayo). \textit{Dimensiones de un cilindro para que el material sea mínimo | La Prof Lina M3} [Vídeo]. YouTube. \url{https://www.youtube.com/watch?v=TVLb1VZUt4c}.
 \bibitem{Stewart} Stewart, J. (2012). \textit{Cálculo de varias variables trascendentes tempranas}. (7th ed.). Cengage Learning Editores. \url{https://intranetua.uantof.cl/estudiomat/calculo3/stewart.pdf}.
\end{thebibliography}

\end{document}